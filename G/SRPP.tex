\chapter{Security Response Plan Policy}
\CommonIntroduction
\section{Overview}
A \acrfull{srp} provides the impetus for security and business teams to integrate their efforts from the perspective of awareness and communication, as well as coordinated response in times of crisis (security vulnerability identified or exploited).  
Specifically, an \gls{srp} defines a product description, contact information, escalation paths, expected \gls{sla}, severity and impact classification, and mitigation/remediation timelines.  
By requiring business units to incorporate an \gls{srp} as part of their business continuity operations and as new products or services are developed and prepared for release to consumers, ensures that when an incident occurs, swift mitigation and remediation ensues.%TODO Wordy, pompus
\section{Purpose}
The purpose of this policy is to establish the requirement that all business units supported by the \gls{infosec} team develop and maintain a security response plan.  
This ensures that security incident management team has all the necessary information to formulate a successful response should a specific security incident occur.
\section{Scope}
This policy applies any established and defined business unity or entity within the \CompanyName{}.
\section{Policy}
The development, implementation, and execution of a \acrfull{srp} are the primary responsibility of the specific business unit for whom the \gls{srp} is being developed in co\"{o}peration with the \gls{infosec} team.  
Business units are expected to properly facilitate the \gls{srp} for applicable to the service or products they are held accountable.  
The business unit security coordinator or champion is further expected to work with \CompanyName{} \gls{infosec} unit in the development and maintenance of an \gls{srp}.
\subsection{Service or Product Description}
The product description in an \gls{srp} must clearly define the service or application to be deployed with additional attention to data flows, logical diagrams, \ins{and/or} architecture considered highly useful.
\subsection{Contact Information}
The \gls{srp} must include contact information for dedicated team members to be available during non-business hours should an incident occur and escalation be required.  
This may be a 24/7 requirement depending on the defined business value of the service or product, coupled with the impact to customer.  
The \gls{srp}\del{ document} must include all phone numbers and \email{} addresses for the dedicated team member(s).
\subsection{Triage}
The \gls{srp} must define triage steps to be coordinated with the security incident management team in a co\"{o}perative manner with the intended goal of swift security vulnerability mitigation.  
This step typically includes validating the reported vulnerability or compromise.
\subsection{Identified Mitigations and Testing}
The \gls{srp} must include a defined process for identifying and testing mitigations prior to deployment.  
These details should include both short-term mitigations as well as the remediation process.
\subsection{Mitigation and Remediation Timelines}
The \gls{srp} must include levels of response to identified vulnerabilities that define the expected timelines for repair based on severity and impact to consumer, brand\oxford{} and company.  
These response guidelines should be carefully mapped to level of severity determined for the reported vulnerability.
\section{Policy Compliance}
\subsection{Compliance Measurement}
Each business unit must be able to demonstrate they have a written \gls{srp} in place, \del{and }that it is under version control\ins{\oxford{}} and is available \del{via the web}.  
The policy should be reviewed annually.
\subsection{Exceptions}
Any exception to this policy must be approved by the \gls{infosec} team in advance and have a written record.
\subsection{Non-Compliance}
Any business unit found to have violated \oldnew{(no SRP developed prior to service or product deployment) this policy}{this policy, such as failing to develop an \gls{srp} prior to deployment of a service or product, }may be subject to delays in service or product release until such a time as the \gls{srp} is developed and approved.  
Responsible parties may be subject to disciplinary action, up to and including termination of employment, should a security incident occur in the absence of an \gls{srp}.
\section{Related Standards, Policies and Processes}
None. 
\section{Definitions and Terms}
None.
\section{Revision History}
\begin{tabular}{|p{1.25in}|p{1.25in}|p{3in}|}
\hline
	Date of Change&
	Responsible&
	Summary of Change\\
\hline
	June 2014&
	SANS Policy Team&
	Separated out from the Password Policy and converted to new format.\\
\hline
	Dec. 2016\newline{}Jan. 2017&
	\xio{}&
	Conversion to \LaTeX{}.\\
\hline
	 &
	 &
	 \\
\hline
\end{tabular}
