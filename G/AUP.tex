\chapter{Acceptable Use Policy}\label{G:AUP}
\CommonIntroduction

\section{Overview}
\gls{infosec}'s intentions for publishing an \textsl{Acceptable Use Policy} are not to impose restrictions that are contrary to \CompanyName{}'s established culture of openness, trust and integrity.  
\gls{infosec} is committed to protecting \CompanyName{}'s employees, partners and the company from illegal or damaging actions by individuals, either knowingly or unknowingly.  

Internet/Intranet/Extranet-related systems, including but not limited to computer equipment, software, operating systems, storage media, network accounts providing electronic mail, WWW browsing, and \gls{ftp}, are the property of \CompanyName{}.  
These systems are to be used for business purposes in serving the interests of the company, and of our clients and customers in the course of normal operations.  
Please review \textsl{Human Resources} policies for further details.

Effective security is a team effort involving the participation and support of every \CompanyName{} employee and affiliate who deals with information and/or information systems. It is the responsibility of every computer user to know these guidelines, and to conduct their activities accordingly.

\section{Purpose}
The purpose of this policy is to outline the acceptable use of computer equipment at \CompanyName{}.  
These rules are in place to protect the employee and \CompanyName{}.  
Inappropriate use exposes \CompanyName{} to risks including virus attacks, compromise of network systems and services, and legal issues.  

\section{Scope}
This policy applies to the use of information, electronic and computing devices, and network resources to conduct \CompanyName{} business or interact with internal networks and business systems, whether owned or leased by \CompanyName{}, the employee, or a third party.  
All employees, contractors, consultants, temporary, and other workers at \CompanyName{} and its subsidiaries are responsible for exercising good judgment regarding appropriate use of information, electronic devices, and network resources in accordance with \CompanyName{} policies and standards, and local laws and regulation.
Exceptions to this policy are documented in \see{G:AUP:PC:E}.

This policy applies to employees, contractors, consultants, temporaries, and other workers at \CompanyName{}, including all personnel affiliated with third parties.  
This policy applies to all equipment that is owned or leased by \CompanyName{}.

\section{Policy}
\subsection{General Use and Ownership}
\begin{enumerate}
\item
\CompanyName{} proprietary information stored on electronic and computing devices whether owned or leased by \CompanyName{}, the employee or a third party, remains the sole property of \CompanyName{}.  
You must ensure through legal or technical means that proprietary information is protected in accordance with the \textsl{Data Protection Standard}.
\item
You have a responsibility to promptly report the theft, loss or unauthorized disclosure of \CompanyName{} proprietary information.
\item
You may access, use or share \CompanyName{} proprietary information only to the extent it is authorized and necessary to fulfill your assigned job duties.
\item
Employees are responsible for exercising good judgment regarding the reasonableness of personal use.  
Individual departments are responsible for creating guidelines concerning personal use of Internet/Intranet/Extranet systems.  
In the absence of such policies, employees should be guided by departmental policies on personal use, and if there is any uncertainty, employees should consult their supervisor or manager.
\item
For security and network maintenance purposes, authorized individuals within \CompanyName{} may monitor equipment, systems and network traffic at any time, per \gls{infosec}'s \textsl{Audit Policy}.%TODO
\item
\CompanyName{} reserves the right to audit networks and systems on a periodic basis to ensure compliance with this policy.
\end{enumerate}
\subsection{Security and Proprietary Information}
\begin{enumerate}
\item
All mobile and computing devices that connect to the internal network must comply with the \textsl{Minimum Access Policy}.%TODO
\item
System level and user level passwords must comply with the \textsl{Password Policy}.  
Providing access to another individual, either deliberately or through failure to secure its access, is prohibited.
\item
All computing devices must be secured with a password-protected screensaver with the automatic activation feature set to 10 minutes or less.  
You must lock the screen or log off when the device is unattended.
\item
\oldnew{Postings by employees from a \CompanyName{} email address to newsgroups should contain a disclaimer stating that the opinions expressed are strictly their own and not necessarily those of \CompanyName{}, unless posting is in the course of business duties.}{Employees are prohibited from using a \CompanyName{} \email{} address to post messages to newsgroups, bulletin boards\oxford{} or social media, unless posting is in the course of \CompanyName{} business duties.}
\item
Employees must use extreme caution when opening \email{} attachments received from unknown senders, which may contain malware.
\end{enumerate}

\subsection{Unacceptable Use}
The following activities are, in general, prohibited.  
Employees may be exempted from these restrictions during the course of their legitimate job responsibilities (e.g., systems administration staff may have a need to disable the network access of a host if that host is disrupting production services).

Under no circumstances is an employee of \CompanyName{} authorized to engage in any activity that is illegal under local, state, federal or international law while utilizing \CompanyName{}-owned resources.  

The lists below are by no means exhaustive, but attempt to provide a framework for activities which fall into the category of unacceptable use.
\subsubsection{System and Network Activities}
The following activities are strictly prohibited, with no exceptions:
\begin{enumerate}
\item
Violations of the rights of any person or company protected by copyright, trade secret, patent or other intellectual property, or similar laws or regulations, including, but not limited to, the installation or distribution of \q{pirated} or other software products that are not appropriately licensed for use by \CompanyName{}.
\item
Unauthorized copying of copyrighted material including, but not limited to, digitization and distribution of photographs from magazines, books or other copyrighted sources, copyrighted music\oxford{} and the installation of any copyrighted software for which \CompanyName{} or the end user does not have an active license is strictly prohibited.
\item
Accessing data, a server or an account for any purpose other than conducting \CompanyName{} business, even if you have authorized access, is prohibited.
\item
Exporting software, technical information, encryption software\oxford{} or encryption technology, in violation of international or regional export control laws, is illegal.  
The appropriate management should be consulted prior to export of any material that is in question.
\item
Introduction of malicious programs into the network or server (e.g., viruses, worms, Trojan horses, \email{} bombs, \etc{}.)
\item
Revealing your account password to others or allowing use of your account by others.  
This includes family and other household members when work is being done at home.
\item
Using a \CompanyName{} computing asset to actively engage in procuring or transmitting material that is in violation of sexual harassment or hostile workplace laws in the user's local jurisdiction.
\item
Making fraudulent offers of products, items, or services originating from any \CompanyName{} account.
\item
Making statements about warranty, expressly or implied, unless it is a part of normal job duties.
\item
Effecting security breaches or disruptions of network communication.  
Security breaches include, but are not limited to, accessing data of which the employee is not an intended recipient or logging into a server or account that the employee is not expressly authorized to access, unless these duties are within the scope of regular duties.  
For purposes of this section, \q{disruption} includes, but is not limited to, network sniffing, pinged floods, packet spoofing, denial of service\oxford{} and forged routing information for malicious purposes.
\item
Port scanning or security scanning is expressly prohibited unless prior notification to \gls{infosec} is made.
\item
Executing any form of network monitoring which will intercept data not intended for the employee's host, unless this activity is a part of the employee's normal job/duty.
\item
Circumventing user authentication or security of any host, network\oxford{} or account.
\item
Introducing honeypots, honeynets\oxford{} or similar technology on the \CompanyName{} network.
\item
Interfering with or denying service to any user other than the employee's host (for example, denial of service attack).
\item
Using any program/script/command, or sending messages of any kind, with the intent to interfere with, or disable, a user's terminal session, via any means, locally or via the Internet/Intranet/Extranet.
\item
Providing information about, or lists of, \CompanyName{} employees to parties outside \CompanyName{}.
\end{enumerate}

\subsection{\Email{} and Communication Activities}
When using company resources to access and use the Internet, users must realize they represent the company.  
Whenever employees state an affiliation to the company, they must also clearly indicate that \q{the opinions expressed \oldnew{are my own and}{do} not necessarily \oldnew{those of the company}{reflect the opinion of \CompanyName{}}.}  
Questions may be addressed to the IT Department.
\begin{enumerate}
\item
Sending unsolicited \email{} messages, including the sending of \q{junk mail}, or other advertising material to individuals who did not specifically request such material (\email{} spam).
\item
Any form of harassment via \email{}, telephone\oxford{} or paging, whether through language, frequency\oxford{} or size of messages.
\item
Unauthorized use, or forging, of \email{} header information.
\item
Solicitation of \email{} for any other \email{} address, other than that of the poster's account, with the intent to harass or to collect replies.
\item
Creating or forwarding \q{chain letters}, \q{Ponzi}\oxford{} or other \q{pyramid} schemes of any type.
\item
Use of unsolicited \email{} originating from within \CompanyName{}'s networks of other Internet/Intranet/Extranet service providers on behalf of, or to advertise, any service hosted by \CompanyName{} or connected via \CompanyName{}'s network. 
\item
Posting the same or similar non-business-related messages to large numbers of Usenet newsgroups (newsgroup spam).
\end{enumerate}

\subsection{Blogging and Social Media}
\begin{enumerate}
\item
Blogging by employees, whether using \CompanyName{}'s property and systems or personal computer systems, is also subject to the terms and restrictions set forth in this Policy.  
Limited and occasional use of \CompanyName{}'s systems to engage in blogging is acceptable, provided that it is done in a professional and responsible manner, does not otherwise violate \CompanyName{}'s policy, is not detrimental to \CompanyName{}'s best interests\oxford{} and does not interfere with an employee's regular work duties.  
Blogging from \CompanyName{}'s systems is also subject to monitoring.
\item
\CompanyName{}'s \textsl{Confidential Information policy} also applies to blogging.  
As such, employees are prohibited from revealing any \CompanyName{} confidential or proprietary information, trade secrets\oxford{} or any other material covered by \CompanyName{}'s \textsl{Confidential Information policy} when engaged in blogging.
\item
Employees shall not engage in any blogging that may harm or tarnish the image, reputation\oxford{} and/or goodwill of \CompanyName{} and/or any of its employees.  
Employees are also prohibited from making any discriminatory, disparaging, defamatory\oxford{} or harassing comments when blogging or otherwise engaging in any conduct prohibited by \CompanyName{}'s \textsl{Non-Discrimination and Anti-Harassment policy}.
\item
Employees may also not attribute personal statements, opinions\oxford{} or beliefs to \CompanyName{} when engaged in blogging.  
If an employee is expressing his or her beliefs and/or opinions in blogs, the employee may not, expressly or implicitly, represent themselves as an employee or representative of \CompanyName{}.  
Employees assume any and all risk associated with blogging.
\item
Apart from following all laws pertaining to the handling and disclosure of copyrighted or export controlled materials, \CompanyName{}'s trademarks, logos\oxford{} and any other \CompanyName{} intellectual property may also not be used in connection with any blogging activity.
\end{enumerate}



%	Internal link
\section{Policy Compliance}
\subsection{Compliance Measurement}
The \gls{infosec} team will verify compliance to this policy through various methods, including but not limited to, periodic walk-throughs, video monitoring, business tool reports, internal and external audits, and feedback to the policy owner.
\subsection{Exceptions}\label{G:AUP:PC:E}
Any exception to the policy must be approved by the \gls{infosec} team in advance.
\subsection{Non-Compliance}
An employee found to have violated this policy may be subject to disciplinary action, up to and including termination of employment.

\section{Related Standards, Policies and Processes}
\begin{itemize}
\item Data Classification Policy
\item Data Protection Standard
\item Social Media Policy
\item Minimum Access Policy
\item \oldnew{Password Policy}{\hyperref[G:PPP]{Password Protection Policy}}
\end{itemize}


\section{Definitions and Terms}
The following definition and terms can be found in the \href{https://www.sans.org/security-resources/glossary-of-terms/}{SANS Glossary} located at \url{https://www.sans.org/security-resources/glossary-of-terms/}
\begin{itemize}
\item Blogging
\item Honeypot
\item Honeynet
\item Proprietary Information
\item Spam
\end{itemize}

\CommonRevisionHistory