\chapter{Disaster Recovery Plan Policy}
\CommonIntroduction
\section{Overview}
Since disasters happen so rarely, management often ignores the disaster recovery planning process.  
It is important to realize that having a contingency plan in the event of a disaster gives \CompanyName{} a competitive advantage.  
This policy requires management to financially support and diligently attend to disaster contingency planning efforts.  
Disasters are not limited to adverse weather conditions.  
Any event that could likely cause an extended delay of service should be considered.  
\oldnew{The}{A} \oldnew{D}{d}isaster \oldnew{R}{r}ecovery \oldnew{P}{p}lan is often part of \oldnew{the}{a} \oldnew{B}{b}usiness \oldnew{C}{c}ontinuity \oldnew{P}{p}lan.
\section{Purpose}
This policy defines the requirement for a baseline disaster recovery plan to be developed and implemented by \CompanyName{} that will describe the process to recover \gls{it} \oldnew{S}{s}ystems, \oldnew{A}{a}pplications\oxford{} and \oldnew{D}{d}ata from any type of disaster that causes a major outage.
\section{Scope}
This policy is directed to the \gls{it} Management Staff who is\ins{/are} accountable to ensure the plan is developed, tested\oxford{} and kept up-to-date.  
This policy is solely to state the requirement to have a disaster recovery plan\oldnew{,}{ -} it does not provide requirement\oldnew{ around}{s regarding} what goes into the plan or sub-plans.
\section{Policy}
\subsection{Contingency Plans}
The following contingency plans must be created:
\begin{itemize}
\item{Computer Emergency Response Plan}:
Who is to be contacted, when, and how?  
What immediate actions must be taken in the event of certain occurrences?
\item{Succession Plan}:
Describe the flow of responsibility when normal staff is unavailable to perform their duties.
\item{Data Study}:
Detail the data stored on the systems, its criticality, and its confidentiality.
\item{Criticality of Service List}:
List all the services provided and their order of importance.  
It also explains the order of recovery in both short-term and long-term timeframes.
\item{Data Backup and Restoration Plan}:
Detail which data is backed up, the media to which it is saved, where that media is stored, and how often the backup is done.  
It should also describe how that data could be recovered.
\item{Equipment Replacement Plan}:
Describe what equipment is required to begin to provide services, list the order in which it is necessary, and note where to purchase the equipment.
\item{Mass Media Management}:
Who is in charge of giving information to the mass media?  
Also\ins{,} provide some guidelines on what data is appropriate to be provided.
\end{itemize}
After creating the plans, it is important to practice them to the extent possible.  
Management should set aside time to test implementation of the disaster recovery plan.  
Table top exercises should be conducted annually.  
During these tests, issues that may cause the plan to fail can be discovered and corrected in an environment that has few consequences.

The plan, at a minimum, should be reviewed an updated on an annual basis.
\CommonPolicyCompliance
\section{Related Standards, Policies\oxford{} and Processes}
None.
\section{Definitions and Terms}
\section{Definitions and Terms}
The following definition and terms can be found in the \href{https://www.sans.org/security-resources/glossary-of-terms/}{SANS Glossary} located at \url{https://www.sans.org/security-resources/glossary-of-terms/}
\begin{itemize}
\item
Disaster
\end{itemize}
\CommonRevisionHistory
