\chapter{End User Encryption Key Protection Policy}
\CommonIntroduction
\section{Overview}
Encryption Key Management, if not done properly, can lead to compromise and disclosure of private keys use to secure sensitive data and hence, compromise of the data.  
While users may understand it's important to encrypt\del{ion} certain documents and electronic communications, they may not be familiar with minimum standards for protection encryption keys.
\section{Purpose}
This policy outlines the requirements for protecting encryption keys that are under the control of end users.  
These requirements are designed to prevent unauthorized disclosure and subsequent fraudulent use.  
The protection methods outlined will include operational and technical controls, such as key backup procedures, encryption under a separate key\oxford{} and use of tamper-resistant hardware.
\section{Scope}
This policy applies to any encryption keys listed below and to the person responsible for any encryption key listed below.  
The encryption keys covered by this policy are:
\begin{itemize}
\item encryption keys issued by \CompanyName{}
\item encryption keys used for \CompanyName{} business
\item encryption keys used to protect data owned by \CompanyName{}
\end{itemize}
The public keys contained in digital certificates are specifically exempted from this policy.
\section{Policy}
All encryption keys covered by this policy must be protected to prevent their unauthorized disclosure and subsequent fraudulent use.
\subsection{Secret Key Encryption Keys}
Keys used for secret key encryption, also called symmetric cryptography, must be protected as they are distributed to all parties that will use them.  
During distribution, the symmetric encryption keys must be encrypted using a stronger algorithm with a key of the longest key length for that algorithm authorized in \CompanyName{}'s \oldnew{\textit{Acceptable Encryption Policy}}{\see{G:AEP}}.  
If the keys are for the strongest algorithm, then the key must be split, each portion of the key encrypted with a different key that is the longest key length authorized and the each encrypted portion is transmitted using different transmission mechanisms.  
The goal is to provide more stringent protection to the key than the data that is encrypted with that encryption key.

Symmetric encryption keys, when at rest, must be protected with security measures at least as stringent as the measures used for distribution of that key.
\subsection{Public Key Encryption Keys}
Public key cryptography, or asymmetric cryptography, uses public-private key pairs.  
The public key is passed to the certificate authority to be included in the digital certificate issued to the end user.  
The digital certificate is available to everyone once it issued.  
The private key should only be available to the end user to whom the corresponding digital certificate is issued.
\subsubsection{\CompanyName{}'s \gls{pki} Keys}
The public-private key pairs used by the \CompanyName{}'s \gls{pki} are generated on the tamper-resistant \smartcard{} issued to an individual end user.  
The private key associated with an end user's identity certificate, which are only used for digital signatures, will never leave the \smartcard{}.  
This prevents the \gls{infosec} Team from escrowing any private keys associated with identity certificates.  
The private key associated with any encryption certificates, which are used to encrypt email and other documents, must be escrowed in compliance with \CompanyName{} policies.

Access to the private keys stored on a \CompanyName{}\ins{-}issued \smartcard{} will be protected by a \gls{pin} known only to the individual to whom the \smartcard{} is issued.  
The \smartcard{} software will be configured to require entering the \gls{pin} prior to any private key contained on the \smartcard{} being accessed.
\subsubsection{Other Public Key Encryption Keys}
Other types of keys may be generated in software on the end user's computer and can be stored as files on the hard drive or on a hardware token.  
If the public-private key pair is generated on \smartcard{}, the requirements for protecting the private keys are the same as those for private keys associated with \CompanyName{}'s \gls{pki}.  
If the keys are generated in software, the end user is required to create at least one backup of these keys and store any backup copies securely.  
The user is also required to create an escrow copy of any private keys used for encrypting data and deliver the escrow copy to the local \gls{infosec} representative for secure storage. 

The \gls{infosec} Team shall not escrow any private keys associated with identity certificates.  
All backups, including escrow copies, shall be protected with a password or passphrase that is compliant with \CompanyName{} \oldnew{\textit{Password Policy}}{password policies (see \see{G:PCG} and \see{G:PPP})}.
\gls{infosec}{} representatives will store and protect the escrowed keys as described in the \CompanyName{} \textsl{Certificate Practice Statement Policy}.%TODO Certificate Practice Statement Policy
\paragraph{Commercial or Outside Organization \gls{pki} Keys}
In working with business partners, the relationship may require the end users to use public-private key pairs that are generated in software on the end user's computer.  
In these cases, the public-private key pairs are stored in files on the hard drive of the end user.  
The private keys are only protected by the strength of the password or passphrase chosen by the end user.  
For example, when an end user requests a digital certificate from a commercial \gls{pki}, such as \href{http://www.verisign.com/}{VeriSign} or \href{https://www.thawte.com/}{Thawte}, the end user's web browser will generate the key pair and submit the public key as part of the certificate request to the \gls{ca}.  
The private key remains in the browser's certificate store where the only protection is the password on the browser's certificate store.  
A web browser storing private keys will be configured to require the user to enter the certificate store password anytime a private key is accessed.
\paragraph{\gls{pgp} Key Pairs}
If the business partner requires the use of \gls{pgp}, the public-private key pairs can be stored in the user's key ring files on the computer hard drive or on a hardware token, for example, a USB drive or a \smartcard{}.  
Since the protection of the private keys is the passphrase on the secret keying, it is preferable that the public-private keys are stored on a hardware token.  
\gls{pgp} will be configured to require entering the passphrase for every use of the private keys in the secret key ring.
\subsection{Hardware Token Storage}
Hardware tokens storing encryption keys will be treated as sensitive company equipment, as described in \CompanyName{}'s \textsl{Physical Security policy}, when outside company offices.  %TODO PhySec Pol
In addition, all hardware tokens, \smartcard{}s, USB tokens, \etc, will not be stored or left connected to any end user's computer when not in use.  
For end users traveling with hardware tokens, they will not be stored or carried in the same container or bag as any computer.
\subsection{\gls{pin}s, Passwords\oxford{} and Passphrases}
All \gls{pin}s, passwords\oxford{} or passphrases used to protect encryption keys must meet complexity and length requirements described in \oldnew{\CompanyName{}'s \textit{Password Policy}}{\see{G:PCG}}.
\subsection{Loss and Theft}
The loss, theft, or potential unauthorized disclosure of any encryption key covered by this policy must be reported immediately to the \gls{infosec} Team.  
\gls{infosec} personnel will direct the end user in any actions that will be required regarding revocation of certificates or public-private key pairs.
\CommonPolicyCompliance
\section{Related Standards, Policies\oxford{} and Processes}
\begin{itemize}
\item \see{G:AEP}%Acceptable Encryption Policy
\item Certificate Practice Statement Policy%TODO
\item \del{Password Policy}
\item \ins{\see{G:PCG}}
\item \ins{\see{G:PPP}}
\item Physical Security Policy%TODO
\end{itemize}
\CommonDefinitionsAndTerms
\begin{itemize}
\item \acrfull{ca}
\item Digital certificate
\item Digital signature
\item Key escrow
\item \Plaintext
\item Public key cryptography
\item Public key pairs
\item Symmetric cryptography
\end{itemize}
\CommonRevisionHistory