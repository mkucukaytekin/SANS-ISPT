\chapter{Remote Access Tools Policy}
\CommonIntroduction
\section{Overview}
Remote desktop software, also known as remote access tools, provide a way for computer users and support staff alike to share screens, access work computer systems from home, and vice versa.  
Examples of such software include \href{https://secure.logmein.com/home/en}{LogMeIn}, \href{https://get.gotomypc.com/}{GoToMyPC}, \href{https://en.wikipedia.org/wiki/Virtual_Network_Computing}{\gls{vnc}}, and \href{https://msdn.microsoft.com/en-us/library/aa383015(v=vs.85).aspx}{Windows \gls{rdp}}.  
While these tools can save significant time and money by eliminating travel and enabling collaboration, they also provide a \backdoor{} into the \CompanyName{} network that can be used for theft of, unauthorized access to\oxford{} or destruction of assets.  
As a result, only approved, monitored\oxford{} and properly controlled remote access tools may be used on \CompanyName{} computer systems.

\section{Purpose}
This policy defines the requirements for remote access tools used at \CompanyName{}.
\section{Scope}
This policy applies to all remote access where either end of the communication terminates at a \CompanyName{} computer asset.
\section{Policy}
All remote access tools used to communicate between \CompanyName{} assets and other systems must comply with the following policy requirements.

\subsection{Remote Access Tools}
\CompanyName{} provides mechanisms to collaborate between internal users, with external partners\oxford{} and from non-\CompanyName{} systems.  
The approved software list can be obtained from \oldnew{<link-to-approved-remote-access-software-list>}{the Remote Access Administrator and \gls{infosec}}.  
Because proper configuration is important for secure use of these tools, mandatory configuration procedures are provided for each of the approved tools.

The approved software list may change at any time, but the following requirements will be used for selecting approved products:
\begin{enumerate}
\item
All remote access tools or systems that allow communication to \CompanyName{} resources from the Internet or external partner systems must require multi-factor authentication.  
Examples include authentication tokens and \smartcard{}s that require an additional \gls{pin} or password.
\item
The authentication database source must be \gls{ad} or \gls{ldap}, and the authentication protocol must involve a challenge-response protocol that is not susceptible to replay attacks.  
The remote access tool must mutually authenticate both ends of the session.
\item
Remote access tools must support the \CompanyName{} application layer proxy rather than direct connections through the perimeter firewall(s).
\item
Remote access tools must support strong, end-to-end encryption of the remote access communication channels as specified in the \CompanyName{} network encryption protocols policy.
\item
All \CompanyName{} \gls{av}, \gls{dlprev}\oxford{} and other security systems must not be disabled, interfered with\oxford{} or circumvented in any way.
\end{enumerate}

All remote access tools must be purchased through the standard \CompanyName{} procurement process, and the \gls{it} group must approve the purchase.
\CommonPolicyCompliance
\section{Related Standards, Policies\oxford{} and Processes}
None.
\CommonDefinitionsAndTerms
\begin{itemize}
\item Application Layer Proxy
\end{itemize}
\CommonRevisionHistory